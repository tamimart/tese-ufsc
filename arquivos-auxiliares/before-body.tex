\begin{center}
\thispagestyle{empty}
    
%\includegraphics{img/brasao_UFSC}
UNIVERSIDADE FEDERAL DE SANTA CATARINA\\
\MakeUppercase{$if(teseufsc.centro)$$for(centro)$$centro$$sep$ \\ $endfor$$endif$}
PROGRAMA DE PÓS-GRADUAÇÃO EM \uppercase{$if(teseufsc.PPG)$$for(PPG)$$teseufsc.PPG$$sep$ \\ $endfor$$endif$}
\vspace{6em}

{$if(teseufsc.autor)$$for(autor)$$teseufsc.autor$$sep$ \\ $endfor$$endif$}

\vspace{6em}

\textbf{$if(teseufsc.titulo)$$teseufsc.titulo$$endif$}$if(teseufsc.subtitulo)$: {$teseufsc.subtitulo$}$endif$

\vfill

Florianópolis\\
{if(teseufsc.ano)$$ano$$else$2026$endif$}

\end{center}
\clearpage


%%%%FOLHA DE ROSTO%%%%
\newpage{} \thispagestyle{empty} \clearpage \setcounter{page}{1}
\pagestyle{plain}

\begin{center}

{$if(teseufsc.autor)$$for(author)$$teseufsc.autor$$sep$ \\ $endfor$$endif$}

\vspace{16em}

\textbf{$if(teseufsc.titulo)$$teseufsc.titulo$$endif$}$if(teseufsc.subtitulo)$: {$teseufsc.subtitulo$}$endif$


\end{center}

\vspace{8em}
\singlespacing
\begin{flushright}
\begin{minipage}{0.5\textwidth}
{\fontsize{10}{12}\selectfont
\justifying
\noindent
Tese submetida ao Programa de Pós-Graduação em {$if(teseufsc.PPG)$$for(PPG)$$teseufsc.PPG$$sep$  $endfor$$endif$} da Universidade Federal de Santa Catarina como requisito parcial para a obtenção do título de Doutora em {$if(teseufsc.PPG)$$for(PPG)$$teseufsc.PPG$$sep$ \\ $endfor$$endif$}.\\

\noindent
Orientador (a): Profa. Dra. Cilene Lino de Oliveira\\
Coorientador (a): Dra. Alexandra Bannach-Brown
}
\end{minipage}
\end{flushright}

\onehalfspacing

\vfill

\begin{center}
Florianópolis

{$if(teseufsc.ano)$$for(ano)$$teseufsc.ano$$sep$ \\ $endfor$$endif$}
\end{center}

%%%%FICHA CATALOGRÁFICA%%%%%
\newpage{} \thispagestyle{empty}

\vspace*{\fill}

\setlength{\fboxsep}{10pt}

\noindent\fbox{%
\begin{minipage}{0.9\textwidth}
\begin{center}
Ficha catalográfica para trabalhos acadêmicos

[Insira neste espaço a ficha catalográfica para trabalhos acadêmicos.]

A ficha é elaborada pela autora no seguinte link: \\
\url{http://portalbu.ufsc.br/ficha}
\end{center}
\end{minipage}
}

%%%%FICHA DE CERTIFICAÇÃO%%%%%
\newpage{} \thispagestyle{empty}

\begin{center}

\vspace{1cm}

{$if(teseufsc.autor)$$for(autor)$$teseufsc.autor$$sep$ \\ $endfor$$endif$}

\vspace{1cm}

{$if(teseufsc.autor)$$for(autor)$$teseufsc.autor$$sep$ \\ $endfor$$endif$}

\vspace{1cm}

O presente trabalho em nível {$if(teseufsc.nivel)$$for(nivel)$$teseufsc.nivel$$sep$ \\ $endfor$$endif$} foi avaliado e aprovado, em {$if(teseufsc.dia)$$for(dia)$$teseufsc.dia$$sep$ \\ $endfor$$endif$} de {$if(teseufsc.mes)$$for(mes)$$teseufsc.mes$$sep$ \\ $endfor$$endif$} de {$if(teseufsc.ano)$$for(ano)$$teseufsc.ano$$sep$ \\ $endfor$$endif$}, pela banca examinadora composta pelos seguintes membros:

\vspace{1cm}

Prof(a). [NOME DO(A) EXAMINADOR(A) 1], Dr.(a)

Instituição [nome INSTITUIÇÃO]  

\vspace{1cm}

Prof(a). [NOME DO(A) EXAMINADOR(A) 2], Dr.(a)

Instituição [nome INSTITUIÇÃO]  

\vspace{1cm}

Prof(a). [NOME DO(A) EXAMINADOR(A) 3], Dr.(a)

Instituição [nome INSTITUIÇÃO]  

\vspace{1cm}

Certificamos que esta é a versão original e final do trabalho de conclusão que foi julgado adequado para obtenção do título de Doutora em Farmacologia.

\vspace{2cm}
\noindent\rule{0.6\textwidth}{0.4pt}

Coordenação do Programa de Pós-Graduação

\vspace{2cm}
\noindent\rule{0.6\textwidth}{0.4pt}

Profa. Cilene Lino de Oliveira, Dra.\\
Orientadora
\vfill

Florianópolis, 2026.
\end{center}

%%%%DEDICATÓRIA%%%%%
\newpage{} \thispagestyle{empty}

\begin{flushright}
\vspace*{\fill}
\hspace*{0.5\textwidth}
\begin{minipage}{0.5\textwidth}
{\fontsize{12}{14}\selectfont
\justifying
\noindent
[Dedicatória. Elemento opcional. Se incluída, insira uma dedicatória de forma breve, prestando homenagem ou dedicando seu trabalho.]
}
\end{minipage}
\end{flushright}

%%%%AGRADECIMENTOS%%%%%
\newpage{} \thispagestyle{empty}

\begin{center}

\textbf{AGRADECIMENTOS}

\end{center}

Texto de agradecimentos aos que contribuíram para a realização do trabalho.

%%%%EPÍGRAFE%%%%%
\newpage{} \thispagestyle{empty}

\vspace*{\fill}

\begin{flushright}
\textit{"Frase significativa relacionada ao tema ou à experiência do trabalho."}\\
— Autor da citação
\end{flushright}

%%%%RESUMO%%%%%
\newpage{} \thispagestyle{empty}

\begin{center}

\textbf{RESUMO}

\end{center}

\singlespacing
\noindent 
Testes comportamentais aplicados em roedores são comumente utilizados para predição de efeitos de substâncias potencialmente antidepressivas. O mais utilizado para tal fim é o teste do nado forçado, que desde sua padronização em 1977, vem acumulando publicações com seu uso na literatura. Enquanto a padronização desses testes só foi possível devido ao conhecimento clínico das primeiras substâncias antidepressivas. Esses também auxiliaram na construção dessa literatura, ao apontar novos alvos terapêuticos e levantar possíveis teorias para os mecanismos de ação dos fármacos antidepressivos. Porém, não é sabido o quanto esses mecanismos estão relacionados aos efeitos comportamentais dos antidepressivos. Uma revisão sistemática e meta-análise será conduzida a partir de uma biblioteca de estudos que testaram anidepressivos em roedores submetidos ao teste do nado forçado. Esta biblioteca será atualizada, compreendendo estudos de 1977 até 2023, e a partir do mapeamento desses estudos de acordo com as propriedades experimentais, uma amostra será acessada para elucidar a pergunta de pesquisa. A busca pelos artigos está prevista para ocorrer nas seguintes bases de dados: Medline via Pubmed, Web of Science e EMBASE. Das referências identificadas, está prevista a seleção de acordo com critérios pré-estabelecidos. Os estudos que atenderem aos critérios de elegibilidade passarão pela extração de dados qualitativos e quantitativos, e análise de risco de viés (utilizando a ferramenta Risk of Bias da SYRCLE). Contudo, prevê-se uma grande biblioteca de referências para esse trabalho, do qual demandará de esforço da equipe de trabalho, tempo e habilidades operacionais. Para acelerar os processos de triagem e extração de dados, serão desenvolvidos algoritmos usando técnicas de processamento de linguagem natural para tratamento dos textos, seguido da aplicação de técnicas de aprendizado de máquina e mineração de texto para classificação das referências. Na meta-análise multivariada, será calculado o tamanho de efeito global utilizando o modelo de efeitos aleatórios para os cinco biomarcadores e antidepressivos mais frequentes. A heterogeneidade será estimada por I² e 𝜏². Serão empregados o gráfico em funil, trim-and-fill, Egger's regression e weight-function model para avaliar o risco de viés de publicação. E por fim, será desenvolvido um aplicativo de web que terá módulos para auxiliar farmacologistas em suas revisões e disponibilizar a biblioteca FSTLibrary para posterior exploração por parte dos interessados no assunto.

\vspace{1em}

\noindent \textbf{Palavras-chave:} Eficácia; Biomarcadores; Comportamento; Automatização. 

%%%%ABSTRACT%%%%%
\newpage{} \thispagestyle{empty}

\begin{center}

\textbf{ABSTRACT}

\end{center}

\singlespacing
\justifying

\noindent Versão em inglês do resumo. Contém os mesmos elementos, entre 150 e 500 palavras.

\vspace{1em}

\noindent \textbf{Keywords:} keyword 1; keyword 2; keyword 3.

